\hypertarget{index_intro_sec}{}\section{Introduction}\label{index_intro_sec}
Réalisation d\textquotesingle{}un programme reprenant la méthode décrite dans l\textquotesingle{}article 3 basé sur la double programmation dynamique (pour plus d’information 4). Le threading (enfilage) \mbox{[}1,2,3\mbox{]} est une stratégie pour rechercher des séquences compatibles avec une structure. Seul les carbones α de la protéine seront considérés. Vous utiliserez les potentiels statistiques D\+O\+PE \mbox{[}5\mbox{]}. \hypertarget{index_ressources}{}\section{Ressources}\label{index_ressources}
1) Jones, D.\+T., Taylor, W.\+R. \& Thornton, J.\+M. (1992) A new approach to protein fold recognition. Nature. 358, 86-\/89.~\newline
 2) Jones, D.\+T., Miller, R.\+T. \& Thornton, J.\+M. (1995) Successful protein fold recognition by optimal sequence threading validated by rigorous blind testing. Proteins. 23, 387-\/397.~\newline
 3) Jones, D.\+T. (1998) T\+H\+R\+E\+A\+D\+ER \+: Protein Sequence Threading by Double Dynamic Programming. (in) Computational Methods in Molecular Biology. Steven Salzberg, David Searls, and Simon Kasif, Eds. Elsevier Science. Chapter 13.~\newline
 4) Protein Structure Comparison Using S\+AP -\/ Springer~\newline
 5) \href{http://www.dsimb.inserm.fr/~gelly/doc/dope.par}{\tt http\+://www.\+dsimb.\+inserm.\+fr/$\sim$gelly/doc/dope.\+par} 